\begin{problem}[A\&P 4.11]
  The system
  \begin{subequations}
    \begin{align}
      \label{E:momentum_1}
      M \mb q' &= \mb p, \\
      \label{E:momentum_2}
      \mb p' &= \mb f(\mb q)
    \end{align}
  \end{subequations}
  is in \emph{partitioned form}. It is also a Hamiltonian system with a separable Hamiltonian, i.e., the ODE for $\mb q$ depends only on $\mb p$ and the ODE for $\mb p$ depends only on $\mb q$. This can be used to design special discretizations. Consider a constant step size $h$.
  \begin{enumerate}[(a)]
    \item The \emph{symplectic Euler} method applies backward Euler to \eqref{E:momentum_1} and the forward Euler to \eqref{E:momentum_2}. Show that the resulting method is explicit and first-order accurate.
    \item The \emph{leapfrog}, or \emph{Verlet}, method can be viewed as a staggered midpoint discretization:
    \begin{align*}
      M(\mb q_{n+1/2} - \mb q_{n-1/2}) &= h \mb p_n, \\
      \mb p_n - \mb p_{n-1} &= h \mb f(\mb q_{n-1/2});
    \end{align*}
    i.e., the mesh on which the $\mb q$-approximations ``live'' is staggered by half a step compared to the $\mb p$-mesh. The method can be kick-started by
    \[
      \mb q_{1/2} = \mb q_0 + h/2 M^{-1}\mb p_0.
    \]
    To evaluate $\mb q_n$ at any mesh point, the expression
    \[
      \mb q_n = \frac12 (\mb q_{n-1/2} + \mb q_{n+1/2})
    \]
    can be used.
    
    Show that this method is explicit and second order accurate.
    \item The Morse problem is 
    \[
      M \mb q'' = \mb f(\mb q), \quad \text{ with } \mb f(\mb q) = -\mb \nabla U(\mb q)
    \]
    where $\mb q = q(t)$ is a scalar, $U(q) = D(1 - e^{-S(q-q_0)})^2$, and we use the constants $D = 90.5*.4184e-3$, $S = 1.814$, $q_0 = 1.41$, and $M = 0.9953$.
    
    Integrate the Morse problem using 1000 uniform steps $h$. Apply three methods: forward Euler, symplectic Euler, and leapfrog. Try the values $h = 2$, $h = 2.3684$, and $h = 2.3685$ and plot in each case the discrepancy in the Hamiltonian (which equals 0 for the exact solution).
    
    What are your observations?
  \end{enumerate}
\end{problem}

\FloatBarrier

\begin{solution}
  \begin{enumerate}[(a)]
    \item If we apply the symplectic Euler method we get
    \begin{align*}
      \mb p_{n+1} &= \mb p_n + h \mb f(\mb q_n) \\
      \mb q_{n+1} &= \mb q_n + h M^{-1} \mb p_{n+1}.
    \end{align*}
    Finding $\mb p_{n+1}$ is clearly explicit, and substituting into $\mb q_{n+1}$ we get
    \[
      \mb q_{n+1} = \mb q_n + h M^{-1}(\mb p_n + h \mb f(\mb q_n))
    \]
    which is explicit as well.
  
    We can write this method in vector form as 
    \[
      \begin{pmatrix}
        \mb q_{n+1} \\ \mb p_{n+1}
      \end{pmatrix}
      =
      \begin{pmatrix}
        \mb q_n + h M^{-1} \mb p_{n+1} \\
        \mb p_n + h \mb f(\mb q_n)
      \end{pmatrix}.
    \]
    To simplify a bit, write
    \[
      \mb z_n = 
      \begin{pmatrix}
        \mb q_n \\ \mb p_n
      \end{pmatrix},
      \quad\text{ and }\quad 
      M^{-1} \mb p = \mb g(\mb p).
    \]
    Then the local truncation error is given by
    \begin{align*}
      &\mb z(t_{n+1}) - \mb z(t_n) - h
      \begin{pmatrix}
        M^{-1}\mb p(t_{n+1}) \\
        \mb f(\mb q(t_n))
      \end{pmatrix} \\
      =\, & \mb z(t_n) + h\mb z'(t_n) + \frac{h^2}{2}\mb z''(t_n) + \mb O(h^3) - \mb z(t_n) - h
      \begin{pmatrix}
        M^{-1}\mb p(t_{n+1}) \\
        \mb f(\mb q(t_n))
      \end{pmatrix} \\
      =\, & h \mb z'(t_n) + \frac{h^2}{2}\mb z''(t_n) + \mb O(h^3) - h
      \begin{pmatrix}
        M^{-1}\mb p(t_{n+1}) \\
        \mb f(\mb q(t_n))
      \end{pmatrix} \\
      =\, & h
      \begin{pmatrix}
        \mb q'(t_n) \\
        \mb p'(t_n)
      \end{pmatrix} + \frac{h^2}{2}
      \begin{pmatrix}
        \mb q''(t_n) \\
        \mb p''(t_n)
      \end{pmatrix} - h
      \begin{pmatrix}
        M^{-1}\mb p(t_{n+1}) \\
        \mb f(\mb q(t_n))
      \end{pmatrix} + \mb O(h^3) \\
      =\, & h
      \begin{pmatrix}
        M^{-1}\mb p(t_n) \\
        \mb f(\mb q(t_n))
      \end{pmatrix} + \frac{h^2}{2}
      \begin{pmatrix}
        M^{-1} \mb p'(t_n) \\
        \frac{\partial \mb f}{\partial \mb q}(\mb q(t_n))\mb q'(t_n)
      \end{pmatrix} - h
      \begin{pmatrix}
        M^{-1}(\mb p(t_n) + h \mb p'(t_n) + \mb O(h^2)) \\
        \mb f(\mb q(t_n))
      \end{pmatrix} + \mb O(h^3) \\
      =\, & h
      \begin{pmatrix}
        M^{-1}\mb p(t_n) \\
        \mb f(\mb q(t_n))
      \end{pmatrix} + \frac{h^2}{2}
      \begin{pmatrix}
        M^{-1} \mb p'(t_n) \\
        \frac{\partial \mb f}{\partial \mb q}(\mb q(t_n))\mb q'(t_n)
      \end{pmatrix} - h
      \begin{pmatrix}
        M^{-1}\mb p(t_n) \\
        \mb f(\mb q(t_n))
      \end{pmatrix} + 
      \begin{pmatrix}
        h^2 M^{-1} \mb p'(t_n) + \mb O(h^3) \\
        0
      \end{pmatrix} +
      \mb O(h^3) \\
      =\, & \frac{h^2}{2}
      \begin{pmatrix}
        -M^{-1} \mb p'(t_n) \\
        \frac{\partial \mb f}{\partial \mb q}(\mb q(t_n))\mb  g(\mb p(t_n))
      \end{pmatrix}
      + \mb O(h^3).
    \end{align*}
    Since the local truncation error is order $h^2$, the global error is order $h$, so the method is first order.
    \item For the leapfrog method, we start out knowing $\mb p_0$, $\mb q_0$, and $\mb q_{1/2}$. After that we have
    \begin{align*}
      M(\mb q_{n+1/2} - \mb q_{n-1/2}) &= h \mb p_n, \\
      \mb p_n - \mb p_{n-1} &= h \mb f(\mb q_{n-1/2}).
    \end{align*}
    Assume that we have already computed $\mb p_0, \dots, \mb p_{n-1}$ and $\mb q_{1/2}, \dots, \mb q{n-1/2}$. Then we get $\mb p_n$ via 
    \[
      \mb p_n = \mb p_{n-1} + h \mb f(\mb q_{n-1/2}),
    \]
    and we get $\mb q_{n+1/2}$ via
    \[
      \mb q_{n+1/2} = \mb q_{n-1/2} + h M^{-1} \mb p_n.
    \]
    After that, we can compute $\mb q_n$ directly. Thus, this method is explicit.
    
    Notice that 
    \begin{align*}
      \mb q_n - \mb q_{n-1}
      &= \frac12(\mb q_{n-1/2} + \mb q_{n+1/2}) - \frac12(\mb q_{n-3/2} + \mb q_{n-1/2}) \\
      &= \frac12(\mb q_{n+1/2} - \mb q_{n-3/2}) \\
      &= \frac12(\mb q_{n-1/2} + hM^{-1}\mb p_n - \mb q_{n-3/2}) \\
      &= \frac12(hM^{-1} \mb p_n + (\mb q_{n-1/2} - \mb q_{n-3/2})) \\
      &= \frac12(hM^{-1} \mb p_n + hM^{-1} \mb p_{n-1}) \\
      &= h\left(\frac{1}{2M}(\mb p_n + \mb p_{n-1})\right).
    \end{align*}
    So, we can write the method as 
    \[
      \begin{pmatrix}
        \mb q_n \\ \mb p_n
      \end{pmatrix} = 
      \begin{pmatrix}
        \mb q_{n-1} \\ \mb p_{n-1}
      \end{pmatrix} + h
      \begin{pmatrix}
        \frac{1}{2M}(\mb p_n + \mb p_{n-1}) \\
        \mb f(\mb q_{n-1/2})
      \end{pmatrix}.
    \]
    Next we compute the local truncation error. Note the index shift to make the Taylor series a bit nicer.
    \begin{align*}
      &
      \begin{pmatrix}
        \mb q(t_{n+1}) - \mb q(t_n) - \frac{h}{2M}(\mb p(t_{n+1}) + \mb p(t_n)) \\[0.3em]
        \mb p(t_{n+1}) - \mb p(t_n) - h \mb f(\mb q(t_{n+1/2}))
      \end{pmatrix} \\
      =& 
      \begin{pmatrix}
        \mb q(t_{n+1}) - \mb q(t_n) - \frac{h}{2M}(M \mb q'(t_{n+1}) + M \mb q'(t_n)) \\[0.3em]
        \mb p(t_{n+1}) - \mb p(t_n) - h \mb p'(t_{n+1/2})
      \end{pmatrix} \\
      =& 
      \begin{pmatrix}
        \mb q(t_{n+1}) - \mb q(t_n) - \frac{h}{2}(\mb q'(t_{n+1}) + \mb q'(t_n)) \\[0.3em]
        \mb p(t_{n+1}) - \mb p(t_n) - h \mb p'(t_{n+1/2})
      \end{pmatrix} \\
      =& 
      \begin{pmatrix}
        \mb q(t_n) + h \mb q'(t_n) + \frac{h^2}{2} \mb q''(t_n) + \frac{h^3}{3} \mb q'''(t_n) + O(h^4) - \mb q(t_n) - \frac{h}{2}(\mb q'(t_{n+1}) + \mb q'(t_n)) \\[0.3em]
        \mb p(t_n) + h \mb p'(t_n) + \frac{h^2}{2} \mb p''(t_n) + \frac{h^3}{3} \mb p'''(t_n) + O(h^4) - \mb p(t_n) - h \mb p'(t_{n+1/2})
      \end{pmatrix} \\
      =& 
      \begin{pmatrix}
        \frac{h}{2} \mb q'(t_n) + \frac{h^2}{2} \mb q''(t_n) + \frac{h^3}{3} \mb q'''(t_n) - \frac{h}{2}\mb q'(t_{n+1}) + O(h^4) \\[0.3em]
        h \mb p'(t_n) + \frac{h^2}{2} \mb p''(t_n) + \frac{h^3}{3} \mb p'''(t_n) - h \mb p'(t_{n+1/2}) + O(h^4)
      \end{pmatrix} \\
      =& 
      \begin{pmatrix}
        \frac{h}{2} \mb q'(t_n) + \frac{h^2}{2} \mb q''(t_n) + \frac{h^3}{3} \mb q'''(t_n) - \frac{h}{2}\left(\mb q'(t_n) + h \mb q''(t_n) + \frac{h^2}{2} \mb q'''(t_n) + O(h^3)\right) + O(h^4) \\[0.3em]
        h \mb p'(t_n) + \frac{h^2}{2} \mb p''(t_n) + \frac{h^3}{3} \mb p'''(t_n) - h\left(\mb p'(t_n) + \frac{h}{2} \mb p''(t_n) + \frac{h^2}{8} \mb p'''(t_n) + O(h^3)\right) + O(h^4)
      \end{pmatrix} \\
      =& 
      \begin{pmatrix}
        \frac{h^3}{3} \mb q'''(t_n) - \frac{h^3}{4} \mb q'''(t_n) + O(h^4) \\[0.3em]
        \frac{h^3}{3} \mb p'''(t_n) - \frac{h^3}{8} \mb p'''(t_n) + O(h^4)
      \end{pmatrix}\\
      =& 
      h^3
      \begin{pmatrix}
        \frac{1}{12} \mb q'''(t_n) \\[0.3em]
        \frac{5}{24} \mb p'''(t_n)
      \end{pmatrix} + O(h^4).
    \end{align*}
    Since the local truncation error is order $h^3$, the global error is order $h^2$ and so the method is second order.
    \item 
    \begin{figure}[h!]
      \centering
      \includegraphics[width=0.32\textwidth]{images/01_1_1.pdf}
      \includegraphics[width=0.32\textwidth]{images/01_1_2.pdf}
      \includegraphics[width=0.32\textwidth]{images/01_1_3.pdf}
      \includegraphics[width=0.32\textwidth]{images/01_2_1.pdf}
      \includegraphics[width=0.32\textwidth]{images/01_2_2.pdf}
      \includegraphics[width=0.32\textwidth]{images/01_2_3.pdf}
      \includegraphics[width=0.32\textwidth]{images/01_3_1.pdf}
      \includegraphics[width=0.32\textwidth]{images/01_3_2.pdf}
      \includegraphics[width=0.32\textwidth]{images/01_3_3.pdf}
      \caption{Hamiltonian discrepancy for various methods and step sizes.}
      \label{F:01}
    \end{figure}
    Figure \ref{F:01} shows the Hamiltonian discrepancy for the forward Euler, symplectic Euler, and leapfrog methods using various step sizes. 
    
    We notice that the Hamiltonian discrepancy is higher for the forward Euler and symplectic Euler methods, but it levels off after a short period of time. The Hamiltonian discrepancy is much smaller in the leapfrog method and is periodic. Because of its symplectic nature, the leapfrog method conserves the energy in dynamical systems.
  \end{enumerate}
\end{solution}
