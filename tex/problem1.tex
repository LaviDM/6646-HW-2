\begin{problem}[A\&P 4.11]
  The system
  \begin{subequations}
    \begin{align}
      \label{E:momentum_1}
      M \mb q' &= \mb p, \\
      \label{E:momentum_2}
      \mb p' &= \mb f(\mb q)
    \end{align}
  \end{subequations}
  is in \emph{partitioned form}. It is also a Hamiltonian system with a separable Hamiltonian, i.e., the ODE for $\mb q$ depends only on $\mb p$ and the ODE for $\mb p$ depends only on $\mb q$. This can be used to design special discretizations. Consider a constant step size $h$.
  \begin{enumerate}[(a)]
    \item The \emph{symplectic Euler} method applies backward Euler to \eqref{E:momentum_1} and the forward Euler to \eqref{E:momentum_2}. Show that the resulting method is explicit and first-order accurate.
    \item The \emph{leapfrog}, or \emph{Verlet}, method can be viewed as a staggered midpoint discretization:
    \begin{align*}
      M(\mb q_{n+1/2} - \mb q_{n-1/2}) &= h \mb p_n, \\
      \mb p_n - \mb p_{n-1} &= h \mb f(\mb q_{n-1/2});
    \end{align*}
    i.e., the mesh on which the $\mb q$-approximations ``live'' is staggered by half a step compared to the $\mb p$-mesh. The method can be kick-started by
    \[
      \mb q_{1/2} = \mb q_0 + h/2 M^{-1}\mb p_0.
    \]
    To evaluate $\mb q_n$ at any mesh point, the expression
    \[
      \mb q_n = \frac12 (\mb q_{n-1/2} + \mb q_{n+1/2})
    \]
    can be used.
    
    Show that this method is explicit and second order accurate.
    \item The Morse problem is 
    \[
      M \mb q'' = \mb f(\mb q), \quad \text{ with } \mb f(\mb q) = -\mb \nabla U(\mb q)
    \]
    where $\mb q = q(t)$ is a scalar, $U(q) = D(1 - e^{-S(q-q_0)})^2$, and we use the constants $D = 90.5*.4184e-3$, $S = 1.814$, $q_0 = 1.41$, and $M = 0.9953$.
    
    Integrate the Morse problem using 1000 uniform steps $h$. Apply three methods: forward Euler, symplectic Euler, and leapfrog. Try the values $h = 2$, $h = 2.3684$, and $h = 2.3685$ and plot in each case the discrepancy in the Hamiltonian (which equals 0 for the exact solution). The plot for $h = 2.3684$ is given in Figure 4.7.
    
    What are your observations? [The surprising increase in leapfrog accuracy from $h = 2.3684$ to $h = 2.3685$ relates to a phenomenon called \emph{resonance instability}.]
  \end{enumerate}
\end{problem}

\FloatBarrier

\begin{solution}
  \begin{enumerate}[(a)]
    \item If we apply the symplectic Euler method we get
    \begin{align*}
      \mb p_{n+1} &= \mb p_n + h \mb f(\mb q_n) \\
      \mb q_{n+1} &= \mb q_n + h M^{-1} \mb p_{n+1}.
    \end{align*}
    Finding $\mb p_{n+1}$ is clearly explicit, and substituting into $\mb q_{n+1}$ we get
    \[
      \mb q_{n+1} = \mb q_n + h M^{-1}(\mb p_n + h \mb f(\mb q_n))
    \]
    which is explicit as well.
  
    We can write this method in vector form as 
    \[
      \begin{pmatrix}
        \mb q_{n+1} \\ \mb p_{n+1}
      \end{pmatrix}
      =
      \begin{pmatrix}
        \mb q_n + h M^{-1} \mb p_{n+1} \\
        \mb p_n + h \mb f(\mb q_n)
      \end{pmatrix}.
    \]
    To simplify a bit, write
    \[
      \mb z_n = 
      \begin{pmatrix}
        \mb q_n \\ \mb p_n
      \end{pmatrix},
      \quad\text{ and }\quad 
      M^{-1} \mb p = \mb g(\mb p).
    \]
    Then the local truncation error is given by
    \begin{align*}
      &\mb z(t_{n+1}) - \mb z(t_n) - h
      \begin{pmatrix}
        M^{-1}\mb p(t_{n+1}) \\
        \mb f(\mb q(t_n))
      \end{pmatrix} \\
      =\, & \mb z(t_n) + h\mb z'(t_n) + \frac{h^2}{2}\mb z''(t_n) + \mb O(h^3) - \mb z(t_n) - h
      \begin{pmatrix}
        M^{-1}\mb p(t_{n+1}) \\
        \mb f(\mb q(t_n))
      \end{pmatrix} \\
      =\, & h \mb z'(t_n) + \frac{h^2}{2}\mb z''(t_n) + \mb O(h^3) - h
      \begin{pmatrix}
        M^{-1}\mb p(t_{n+1}) \\
        \mb f(\mb q(t_n))
      \end{pmatrix} \\
      =\, & h
      \begin{pmatrix}
        \mb q'(t_n) \\
        \mb p'(t_n)
      \end{pmatrix} + \frac{h^2}{2}
      \begin{pmatrix}
        \mb q''(t_n) \\
        \mb p''(t_n)
      \end{pmatrix} - h
      \begin{pmatrix}
        M^{-1}\mb p(t_{n+1}) \\
        \mb f(\mb q(t_n))
      \end{pmatrix} + \mb O(h^3) \\
      =\, & h
      \begin{pmatrix}
        M^{-1}\mb p(t_n) \\
        \mb f(\mb q(t_n))
      \end{pmatrix} + \frac{h^2}{2}
      \begin{pmatrix}
        M^{-1} \mb p'(t_n) \\
        \frac{\partial \mb f}{\partial \mb q}(\mb q(t_n))\mb q'(t_n)
      \end{pmatrix} - h
      \begin{pmatrix}
        M^{-1}(\mb p(t_n) + h \mb p'(t_n) + \mb O(h^2)) \\
        \mb f(\mb q(t_n))
      \end{pmatrix} + \mb O(h^3) \\
      =\, & h
      \begin{pmatrix}
        M^{-1}\mb p(t_n) \\
        \mb f(\mb q(t_n))
      \end{pmatrix} + \frac{h^2}{2}
      \begin{pmatrix}
        M^{-1} \mb p'(t_n) \\
        \frac{\partial \mb f}{\partial \mb q}(\mb q(t_n))\mb q'(t_n)
      \end{pmatrix} - h
      \begin{pmatrix}
        M^{-1}\mb p(t_n) \\
        \mb f(\mb q(t_n))
      \end{pmatrix} + 
      \begin{pmatrix}
        h^2 M^{-1} \mb p'(t_n) + \mb O(h^3) \\
        0
      \end{pmatrix} +
      \mb O(h^3) \\
      =\, & \frac{h^2}{2}
      \begin{pmatrix}
        -M^{-1} \mb p'(t_n) \\
        \frac{\partial \mb f}{\partial \mb q}(\mb q(t_n))\mb  g(\mb p(t_n))
      \end{pmatrix}
      + \mb O(h^3).
    \end{align*}
    Since the local truncation error is order $h^2$, the global error is order $h$, so the method is first order.
    \item [todo]
    \item 
    \begin{figure}
      \centering
      \includegraphics[width=0.32\textwidth]{images/01_fe1.pdf}
      \includegraphics[width=0.32\textwidth]{images/01_fe2.pdf}
      \includegraphics[width=0.32\textwidth]{images/01_fe3.pdf}
      \includegraphics[width=0.32\textwidth]{images/01_se1.pdf}
      \includegraphics[width=0.32\textwidth]{images/01_se2.pdf}
      \includegraphics[width=0.32\textwidth]{images/01_se3.pdf}
      \includegraphics[width=0.32\textwidth]{images/01_lf1.pdf}
      \includegraphics[width=0.32\textwidth]{images/01_lf2.pdf}
      \includegraphics[width=0.32\textwidth]{images/01_lf3.pdf}
      \caption{Hamiltonian discrepancy for various methods and step sizes.}
      \label{F:hamiltonian}
    \end{figure}
  \end{enumerate}
\end{solution}
