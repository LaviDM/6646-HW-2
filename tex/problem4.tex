\begin{figure}
  \begin{problem}[A\&P 4.19]
    The modified Kepler problem is a Hamiltonian system, i.e., 
    \[
      \mb q' = H_{\mb p}, \quad \mb p' = -H_{\mb q},
    \]
    with the Hamiltonian
    \[
      H(\mb q, \mb p) = \frac{p_1^2 + p_2^2}{2} - \frac1r - \frac{\alpha}{2r^3},
    \]
    where $r = \sqrt{q_1^2 + q_2^2}$, and we take $\alpha = 0.01$. Clearly, $H' = H_{\mb q} \mb q' + H_{\mb p} \mb p' = 0$, so $H(\mb q(t), \mb p(t)) = H(\mb q(0), \mb p(0))$ for all $t$. We consider simulating this system over a long time interval with a relatively coarse, uniform step size $h$, i.e., $bh \gg 1$. The mere accumulation of local errors may then become a problem. For instance, using the explicit midpoint method with $h = 0.1$ and $b = 500$, the approximate solution for $r$ becomes larger than the exact one by two orders of magnitude.
  
    But some methods perform better than would normally be expected. In Figure \ref{F:kepler} we plot $q_1$ vs. $q_2$ (``phase plane portrait'') for (a) the implicit midpoint method using $h = 0.1$, (b) the classical explicit Runge-Kutta method of order 4 using $h = 0.1$, and (c) the exact solution (or rather, a sufficiently close approximation to it). The initial conditions are
    \[
      q_1(0) = 1 - \beta, \quad q_2(0) = 0, \quad p_1(0) = 0, \quad p_2(0) = \sqrt{(1 + \beta)/(1 - \beta)}
    \]
    with $\beta = 0.6$. Clearly, the midpoint solution with this coarse step size outperforms not only the explicit midpoint method but also the fourth-order method. Even though the pointwise error reaches close to 100\% when $t$ is close to $b$, the midpoint solution lies on a torus, like the exact solution, whereas the RK4 (classical fourth-order) picture is noisy. Thus, we see yet again that truncation error is not everything, even in some nonstiff situations, and the theory in this case must include other aspects.
  
    Integrate these equations using the two methods of Figure \ref{F:kepler} with a constant step size $h = 0.1$ and $h = 0.01$ (four runs in total), monitoring the maximum deviation $|H(\mb q(t), \mb p(t)) - H( \mb q(0),\mb p(0))|$. (This is a simple error indicator which typically underestimates the error in the solution components, and is of interest in its own right.) What are your conclusions?
    
    \centering
    \includegraphics[width=0.5\textwidth]{images/kepler}
    \caption{Modified Kepler problem: Approximate and exact solutions.}
    \label{F:kepler}
  \end{problem}
\end{figure}

\FloatBarrier

\begin{solution}
  
\end{solution}