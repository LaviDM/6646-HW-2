\begin{problem}[A\&P 5.9]
  The following ODE system due to H. Robertson models a chemical reaction system and has been used extensively as a test problem for stiff solvers:
  \begin{align*}
    y_1' &= -\alpha y_1 + \beta y_2 y_3, \\
    y_2' &= \alpha y_1 - \beta y_2 y_3 - \gamma y_2^2, \\
    y_3' &= \gamma y_2^2.
  \end{align*}
  Here $\alpha = 0.04$, $\beta = 1.e + 4$, and $\gamma = 3.e + 7$ are slow, fast, and very fast reaction rates. The starting point is $\mb y(0) = (1,0,0)^T$.
  \begin{enumerate}[(a)]
    \item It is known that this system reaches a steady state, i.e., where $\mb y' = \mb 0$. Show that $\sum_{i=1}^3 y_i(t) = 1,\quad 0 \leq t \leq b$; then find the steady state.
    \item Integrate the system using a nonstiff code with a permissive error tolerance (say $1.e - 2$) for the interval length $b = 3$, just to see how inadequate a nonstiff solver can be. How far is $\mb y(b)$ from the steady state?
    \item The steady state is reached very slowly for this problem. Use a stiff solver to integrate the problem for $b = 1.e + 6$ and plot the solution on a semilog scale in $t$. How far is $\mb y(b)$ from the steady state?
  \end{enumerate}
\end{problem}

\FloatBarrier

\begin{solution}
  \begin{enumerate}[(a)]
    \item 
    Let $f(t) = y_1(t) + y_2(t) + y_3(t)$. Then we know that $f(0) = 1 + 0 + 0 = 1$. Also, 
    \begin{align*}
      f'(t) 
      &= y_1'(t) + y_2'(t) + y_3'(t) \\
      &= (-\alpha y_1(t) + \beta y_2(t) y_3(t)) + (\alpha y_1(t) - \beta y_2(t) y_3(t) - \gamma y_2(t)^2) + (\gamma y_2(t)^2) \\
      &= 0. \\
    \end{align*}
    Since the derivative is always $0$, $f(t)$ must be constant. Thus, $\sum_{i=1}^3 y_i(t) = 1,\quad 0 \leq t \leq b$. To solve for the steady state, we want to know where $\mb y' = \mb 0$, or
    \begin{align*}
      -\alpha y_1 + \beta y_2 y_3 &= 0, \\
      \alpha y_1 - \beta y_2 y_3 - \gamma y_2^2 &= 0, \\
      \gamma y_2^2 &= 0.
    \end{align*}
    We immediately see that $y_2$ and $y_1$ must equal $0$. $y_3$ is undetermined by the system, but we must have $y_3 = 1$ from the previous constraints. So, the steady state is $(0,0,1)^T$.
    \item
    \begin{figure}
      \centering
      \includegraphics[width=0.32\textwidth]{images/06_1.pdf}
      \includegraphics[width=0.32\textwidth]{images/06_2.pdf}
      \includegraphics[width=0.32\textwidth]{images/06_3.pdf}
      \caption{$y_1$ vs. $t$, $y_2$ vs. $t$, and $y_3$ vs. $t$ using an explicit Runge-Kutta (4,5) method (nonstiff)}
      \label{F:06_1}
    \end{figure}
    Figure \ref{F:06_1} shows the solution on $[0,3]$ using a nonstiff solver. In order to achieve the permissive error tolerance of $1.e-2$, this method used 8200 steps with a maximum step size of $4.9106 \times 10^{-4}$.
    
    Since the method is accurate of order 4, it should only require a step size on the order of $0.01^{1/4} \approx 0.3162$. It is clear that the IVP is stiff since it requires a much smaller step size to maintain stability. From the graph of $y_2$ vs. $t$, we can still see some instability, even with such small step sizes.
    
    At $t_f$, we have
    \[
      \mb y(3) = \begin{pmatrix}
        0.92189 \\
        2.4285 \times 10^{-5} \\
        0.078089
      \end{pmatrix}.
    \]
    
    It is clear that the solution is still a long ways off from the steady state of $(0,0,1)^T$.
    \FloatBarrier
    
    \item
    \begin{figure}[h!]
      \centering
      \includegraphics[width=0.32\textwidth]{images/06_4.pdf}
      \includegraphics[width=0.32\textwidth]{images/06_5.pdf}
      \includegraphics[width=0.32\textwidth]{images/06_6.pdf}
      \caption{$y_1$ vs. $t$, $y_2$ vs. $t$, and $y_3$ vs. $t$ using a multistep NDF based solver (stiff)}
      \label{F:06_2}
    \end{figure}
    
    Figure \ref{F:06_2} shows the solution on $[0,1000000]$ using a stiff solver. We can see from the images that $t_f$ is large enough to be close to the steady state. In fact, at $t_f$ we have
    \[
      \mb y(1000000) = \begin{pmatrix}
        0.0020317 \\
        8.1433 \times 10^{-9} \\
        0.99797
      \end{pmatrix}.
    \]
    
    This method performs extremely well, requiring just $145$ steps to achieve an even better error tolerance on a much larger interval. While the minimum step size is $2.2910 \times 10^{-4}$, the average step size is much larger. We even have a step size of $6.5459 \times 10^{4}$ when the system gets close to the steady state.
    
    Looking at the graph, we also do not see the instability in $y_2$ vs. $t$. Because the solver is stiff, it does not need to use extremely small step sizes to achieve absolute stability.
  \end{enumerate}
\end{solution}